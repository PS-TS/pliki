\documentclass[a4paper, 12pt]{amsart}
%\usepackage{amsmath}
\usepackage[T1]{fontenc}
\usepackage{enumitem}
\usepackage{fdsymbol}
\usepackage{algorithm}
\usepackage{algpseudocode}
\usepackage{url}
\usepackage{listings}
\usepackage{color}
\definecolor{dkgreen}{rgb}{0,0.6,0}
\definecolor{gray}{rgb}{0.5,0.5,0.5}
\definecolor{mauve}{rgb}{0.58,0,0.82}
\author{PS}
\title{\'Srodowiska}

\lstset{frame=tb,
  language=C++,
  aboveskip=3mm,
  belowskip=3mm,
  showstringspaces=false,
  columns=flexible,
  basicstyle={\small\ttfamily},
  numbers=none,
  numberstyle=\tiny\color{gray},
  keywordstyle=\color{blue},
  commentstyle=\color{dkgreen},
  stringstyle=\color{mauve},
  breaklines=true,
  breakatwhitespace=true,
  tabsize=3
}

\begin{document}
\maketitle \tableofcontents
\section{Listy i spisy}
Lista zakupów
\begin{itemize}[label=$\heartsuit$]
\item Czekolada.
\item Kawa.
\item Mleko.
\end{itemize}
Lista zakupów
\begin{enumerate}[label=(\Alph*)]
\item słodycze:
\begin{enumerate}[label=(\alph*)]
\item cukierki,
\item d\.zem,
\end{enumerate}
\item napoje:
\begin{enumerate}[label=(\arabic*)]
\item kawa,
\item herbata,
\item coca-cola;
\end{enumerate}
\item warzywa:
\begin{enumerate}[label=\Alph*)]
\item dynia,
\item ogórki,
\end{enumerate}
\item owoce:
\begin{enumerate}[label=\alph*.]
\item banana,
\item jabłka,
\item jagody
\begin{enumerate}[label=\Roman*]
\item borówki,
\item maliny.
\end{enumerate}
\end{enumerate}
\item pizza
\end{enumerate}
\section{Macierz}
$$A= \left(\begin{array}{ccc}
12&3&-10\\
x&15&0\\
2.5&-23&12
\end{array}\right)$$
\section{Tabelka}
\begin{tabular}{|c|c|ll|r|}
\hline 
Numer & Album & \multicolumn{2}{|c|}{Imie i Nazwisko} & Ocena\\
\hline\hline
1&11111&Jan&Kowalski&5 \\
\hline
2&22222&Grzegorz&Brzęczyszczykiewicz&4.5 \\
\hline
3&12345&Piotr&Wiśniewski&2\\
\hline
4&12346&Wojciech&Kowalczyk&2.5 \\
\hline
5&12347&Krystyna&Lewandowska&3 \\
\hline
\end{tabular} 
\section{Kod}
\begin{lstlisting}
function power(x:integer,n:integer):integer;
Var k,a,b:integer;
Begin
k:=n;a:=1;b:=x;
while k>0 do begin{Niezmiennik:x^n=a*b^k}
if k mod 2 = 0 then begin
k:=k/2;
b:=b*b;
end else begin
k:=k=1;
a:=a*b;
end;
end;
power:=a;
End;
\end{lstlisting}
\newpage
\section{Algorytm}
\begin{algorithm}
\caption{Moj algorytm}\label{alg:cap}
\begin{algorithmic}
\Require $n \geq 0$
\Ensure $a = x^n$
\State $k \gets n; a \gets 1; b \gets x$
\While{$k > 0$} {[Niezmennik: $x^n = a * b^k$]}
\If{$k$ jest liczba parzysta}
    \State $k \gets k / 2$
    \State $b \gets b * b$
\ElsIf{$k$ jest liczba nieparzysta}
    \State $k \gets k - 1$
    \State $a \gets a * b$
\EndIf
\EndWhile
\end{algorithmic}
\end{algorithm}
\section{Spis literatury}
\bibliographystyle{plain}
\begin{thebibliography}{10}
\bibitem{Muranova1} Anna Muranova. On the notion of effective impedance.
\textit{Operator and Matrices}, 14(3):723-741, 2020.
\bibitem{Woess} Wolfgang Woess. \textit{Random Walks on Infinite Graphs
and Groups}. Cambridge Tracts in Mathematics. Cambridge University Press, 2000. 
\newline
\url{http://wmii.uwm.edu.pl/~muranova/PU2022-23/presentation4.pdf}
\end{thebibliography}
\end{document}